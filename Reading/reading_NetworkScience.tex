\documentclass[a4paper,11pt]{jsarticle}

% 自作マクロ
\input{mymacro.tex}

% \usepackage[dvipdfmx]{hyperref}
% \usepackage[hypertex]{hyperref}
% \usepackage[dvips]{hyperref}

\usepackage{amsmath}

\begin{document}

\title{ネットワーク科学に関する論文を読んだ記録}
\author{香川渓一郎}
\date{\today}
\maketitle

\setcounter{tocdepth}{1}
\tableofcontents

\newpage

\section{Guo, et al. (2017) day:2020.4.15}
    \subsection{論文情報}
    Guo, Hengdao, et al. "A critical review of cascading failure analysis and modeling of power system." Renewable and Sustainable Energy Reviews 80 (2017): 9-22.
    % 著者名, 論文タイトル, 掲載雑誌, 年, ページ.
    \subsection{何に関する論文で何を示したのか}
    カスケード故障(cascading failure)に関する簡単な概観と関連する文献やモデルを分類した.
    またカスケード故障に関連する様々な特徴を示し,異なるモデル間の比較を行った.
    最後に将来注目すべき内容についても触れている.
    \subsection{先行研究と比べてスゴイこと}
    これまで様々な数理モデルや解析手法が提案されカスケード故障における複雑なメカニズムの理解が進められてきた.
    例えば Papic, et al. (2011), Vaiman, et al. (2012), Biatec, et al. (2016) が挙げられる.
    しかしこれら様々なモデルや解析手法の概要や比較検討を行った研究は少なく,本論文でこれらの研究を簡単に概観し,最新のカスケード故障解析ツールやモデルを普及させ,今後の課題を明らかにした.
    \subsection{論文の核となるモノ}
    以下のように6種類に大別されたモデルについて比較を行う.
    \begin{itemize}
      \item トポロジカルモデル (Topological models)
      \begin{itemize}
        \item 変形トポロジカルモデル (Modified topological models)
        \item 極大流モデル (Maximum flow models)
      \end{itemize}
      \item 確率シミュレーションモデル (Stochastic simulation models)
      \begin{itemize}
        \item PRACTICEモデル
        \item マルコフ鎖モデル (Markov chain models)
      \end{itemize}
      \item 高レベル統計モデル (High-level statistical models)
      \begin{itemize}
        \item CASCADEモデル
        \item 分岐プロセスモデル (Branching process models)
      \end{itemize} 
      \item 動的シミュレーションモデル (Dynamic simulation models)
      \begin{itemize}
        \item OPAモデル
        \item マンチェスターモデル (Manchester models)
        \item COSMICモデル
        \item 多重時間スケール準動的モデル (Multi-timescale quasi-dynamic models)
        \item ASSESSモデル
        \item TRELSSモデル
        \item 動的PRAモデル (Dynamic PRA models)
      \end{itemize}
      \item 相互依存モデル (Interdependent models)
      \begin{itemize}
        \item 複雑なネットワークベースの相互依存モデル (Complex network-based interdependent models)
        \item 相互依存マルコフ鎖モデル (Inter-dependent Markov chain models)
        \item フロッキングペースの階層型サイバー物理モデル (Flocking-besed hierarchical cyber-pysical models)
      \end{itemize}
      \item その他
      \begin{itemize}
        \item ポテンシャルカスケードモデル (Potential cascading models)
        \item 隠れ故障モデル (Hidden failure models)
        \item 歴史のデータをベースとしたモデル
      \end{itemize}
    \end{itemize}
    \subsection{どのような手法で示したのか}
    上記のモデルの長所と短所を挙げて比較した.結論としてカスケード故障時の全てのメカニズムを網羅できるモデルは存在しないと結論された.
    トポロジカルモデルは電力網の脆弱性,構成要素の臨界性,堅牢性,構造的な観点からの解析に使用できる.
    確率シミュレーションモデルは可能な限りの不確実性を組み入れているが,電力システムのダイナミクスをシミュレーションするには至らない.
    高レベル統計モデルはカスケードの詳細を無視することでカスケードの平均電波とブラックアウトの分布サイズを迅速に予測することを実現している.
    動的シミュレーションモデルは比較的高い計算コストでシステムのダイナミクスを捉えることができる.
    相互依存モデルは物理ネットワークとサイバーネットワークの両方を考慮し,相互依存性も組み入れている.
    その他のモデルはカスケード故障メカニズム全体ではなく特定の領域の解析に集中しており補完的なツールとして機能する.
    \subsection{今後の展望や課題}
    カスケード故障の原因は予期せぬ制御不能のものであるため総合的な防止には至らないが,今後原因の理解が深まることで解析モデルのシミュレーションが改善される.
    またカスケード故障時の適合メカニズムの解明も今後の課題である.
    様々なモデルが提唱されているがそのどれもが一部のメカニズムしか捉えられていない.
    従来の電力システムの動的解析ツールは多く存在するが,その多くがカスケード時のシステム全体の構成要素間の相互作用を考慮していない.
    異なるモデルをどのように組み合わせるか,異なる状況やカスケードの段階でどのモデルが適切かを検討することも必要である.
    \subsection{この論文を引用している論文}
    Gholami, et al. (2018) \cite{GholamiETAL2018},
    Tu, et al. (2018) \cite{TuETAL2018},
    Eisenberg (2018) \cite{Eisenberg2018},
    Schauer, et al. (2018) \cite{SchauerETAL2018}.
    \subsection{この論文が引用している主要な先行研究}
    Biatec, et al. (2016) \cite{BialekETAL2016},
    Papic, et al. (2011) \cite{PapicETAL2011},
    Vaiman, et al. (2012) \cite{VaimanETAL2012}.

\newpage

\begin{thebibliography}{99}
  \bibitem{BialekETAL2016}
  Bialek, Janusz, et al. "Benchmarking and validation of cascading failure analysis tools." IEEE Transactions on Power Systems 31.6 (2016): 4887-4900.
  \bibitem{Eisenberg2018}
  Eisenberg, Daniel A. How to think about resilient infrastructure systems. Diss. Arizona State University, 2018.
  \bibitem{GholamiETAL2018}
  Gholami, Amin, et al. "Toward a consensus on the definition and taxonomy of power system resilience." IEEE Access 6 (2018): 32035-32053.
  \bibitem{PapicETAL2011}
  Papic, Milorad, et al. "Survey of tools for risk assessment of cascading outages." 2011 IEEE Power and Energy Society General Meeting. IEEE, 2011.
  \bibitem{SchauerETAL2018}
  Schauer, Stefan, et al. "Conceptual Framework for Hybrid Situational Awareness in Critical Port Infrastructures." International Conference on Critical Information Infrastructures Security. Springer, Cham, 2018.
  \bibitem{TuETAL2018}
  Tu, Haicheng, et al. "Optimal robustness in power grids from a network science perspective." IEEE Transactions on Circuits and Systems II: Express Briefs 66.1 (2018): 126-130.
  \bibitem{VaimanETAL2012}
  Vaiman, Marianna, et al. "Risk assessment of cascading outages: Methodologies and challenges." IEEE Transactions on Power Systems 27.2 (2012): 631.

\end{thebibliography}

\end{document}